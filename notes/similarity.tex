\documentclass[12pt,letterpaper]{article}

% typography
\newcommand{\documentname}{\textsl{Note}}

% math
\newcommand{\transpose}[1]{{#1}^{\mathsf{T}}}
\newcommand{\inverse}[1]{{#1}^{-1}}
\newcommand{\given}{\,|\,}
\newcommand{\dd}{\mathrm{d}}
\newcommand{\normal}{\mathcal{N}}

\begin{document}\sloppy\sloppypar\raggedbottom\frenchspacing

\section*{\raggedright How similar are two noisily observed vectors?}

\noindent
\textbf{David W. Hogg}\\
\textsl{NYU, Flatiron, \& MPIA}

\bigskip

\paragraph{Abstract:}
I present a general (although assumption-laden) method for asking the
question: Are two independently and noisily observed objects in fact
identical? My approach is to turn this question into a parameter
estimation, and then post-process the posterior beliefs about the
estimated parameter. It makes heavy use of Gaussians (quadratic forms
in the log) for their mathematical simplicity.

\bigskip

\paragraph{Introduction:}
Imagine that $y_1$ and $y_2$ are two observed data points. Think of
them as $D$-dimensional vectors in a vector space. They could be two
spectra of two stars (Hawkins's problem), or they could be the
detailed element abundances for two stars (Ness's problem). How can we
tell if the two objects observed are intrinsically identical? Would
they appear identical if we had better data?

One option would be to think of making two hypotheses, to wit, ``H1:
The two objects are identical'', and ``H2: The two objects are
different''. Then we could compute the marginalized likelihoods of
these two hypotheses under some priors over the nuisance parameters of
relevance. That's sensible (and more-or-less what we did in S. M. Oh's
paper).

Another option would be to convert this problem into one of parameter
estimation: That is, compute a marginalized likelihood or marginalized
posterior probability density (pdf) for the \emph{difference} between
the two objects, and see if that likelihood or posterior pdf puts
significant density on zero---on a null or vanishing difference
vector.

For various reasons that are (perhaps) badly justified, we are going
to take the second approach in this \documentname. That is, we are
going to estimate the difference between the objects given their
(presumed noisy) measurements, and find out if that difference is
consistent with zero.

Before we start, one side note: I usually recommend that one just
compute chi-squared between the vectors. This is defined as
\begin{eqnarray}
\chi^2 &=&
\transpose{[y_1 - y_2]}\cdot\inverse{[C_1 + C_2]}\cdot [y_1 - y_2] \quad ,
\end{eqnarray}
where, implicitly, $y_1$ and $y_2$ are $D$-dimensional column vectors,
and $C_1$ and $C_2$ are the $D\times D$ covariance matrices that
express our beliefs about the noise variance on the measurements $y_1$
and $y_2$. In this case (that is, if you just compute $\chi^2$) we say
that the two objects are identical if the $\chi^2$ is below some
threshold. That is simple, easy, and probabilistically
justifiable. The objections are manifold. One is that, as the data get
better (the determinants of the $C_1$ and $C_2$ matrices get smaller),
it becomes increasingly harder to conclude similarity or
identicality. To this I say: \emph{Duh!} That will be true of any
correct method. Another objection is that $\chi^2$ treats all elements
of the data (all directions in the vector space) identically, or at
least weights them by their measurement quality rather than their
\emph{usefulness for determining dissimilarity}. If some dimensions
are better than others for determining identicality, $\chi^2$ is blind
to that. That is a significant objection, and the one I
address---below!---in this \documentname.

\paragraph{Assumptions:}
I am going to make the following assumptions, each one of which is
questionable and unpleasant, but each one of which improves our
computational tractability.
\begin{enumerate}\itemsep0ex
\item \emph{Noise:} Observed vectors $y_1$ and $y_2$ are noisy
  measurements of some latent, unobserved \emph{true} vectors $x_1$
  and $x_2$, and the differences between the observed and true vectors
  are drawn from a $D$-dimensional Gaussian distribution with zero
  mean and known variance tensors $C_1$ and $C_2$, respectively. The
  noise draws are independent (but not identically distributed). There
  are no outliers or non-Gaussianities, and the variance tensors are
  not just known but correct.
\item \emph{Prior:} The true vectors are drawn independently from a
  Gaussian distribution of mean $\mu$ and variance tensor $V$. This
  variance tensor might be low-rank. By some algorithm (PCA or HMF,
  for example), magic, or intuition, the mean vector $\mu$ and
  variance tensor $V$ are known in advance.
\end{enumerate}

The first thing we are going to do is diagonalize the prior variance
tensor $V$. We are going to do this so we develop an approach that is
computationally feasible when $V$ is low-rank, and numerically stable
when $V$ has a bad condition number. We will write
\begin{eqnarray}
V &\equiv& R\cdot\Lambda\cdot\transpose{R} \quad ,
\end{eqnarray}
where $R$ is a $D\times K$ matrix of the $K$ ($K\leq D$) eigenvectors
of $V$ that have positive eigenvalues, and $\Lambda$ is the diagonal
$K\times K$ tensor that has the eigenvalues on its diagonal. By the
assumption that the true vectors are drawn from the Gaussian with this
variance and mean $\mu$, we can (without any loss of generality) write
the \emph{true} vectors $x_1$ and $x_2$ as a sum and difference of
coefficient vectors $a$ and $b$ in the $K$-dimensional space as
follows:
\begin{eqnarray}
x_1 &=& R\cdot[a + b] + \mu \\
x_2 &=& R\cdot[a - b] + \mu \quad ,
\end{eqnarray}
where the $D$-vector $R\cdot a + \mu$ is the mean of $x_1$ and $x_2$,
and $R\cdot b$ is the half-difference between $x_1$ and $x_2$. This is
a change of variables.  In this change of variables, the question of
``are $y_1$ and $y_2$ identical?'' will convert to the question of
``is the posterior pdf for $b$ consistent with $b=0$?''. We need a
posterior pdf for $b$!

\paragraph{Bayesian inference:}
Under these assumptions (above), there is only one thing to do:
Compute the posterior pdf for $b$, marginalizing out $a$, given the
data $y_1$ and $y_2$, and our assumptions (which are detailed, and
include the objects $C_1$, $C_2$, $\mu$, and $V$).
\begin{eqnarray}
p(b\given y_1,y_2) &=& \int p(a,b\given y_1,y_2)\,\dd a \\
p(a,b\given y_1,y_2) &=& \frac{1}{Z}\,p(a,b)\,p(y_1\given a,b)\,p(y_2\given a,b) \\
p(a,b) &=& \normal(a+b\given 0,\Lambda)\,\normal(a-b\given 0,\Lambda) \\
p(y_1\given a,b) &=& \normal(y_1\given R\cdot[a+b]+\mu,C_1) \\
p(y_2\given a,b) &=& \normal(y_2\given R\cdot[a-b]+\mu,C_2), \quad
\end{eqnarray}
where the integral is over all values of $a$, the normalization
constant $Z$ can be ignored for now, and the function $\normal(x\given
m,v)$ is the Gaussian pdf for $x$ given mean $m$ and variance
$v$. Switching to the log (and making use of the magic of Gaussians),
\begin{eqnarray}
\ln p(a,b\given y_1,y_2) &=&
  -\frac{1}{2}\,\transpose{[a+b]}\cdot\inverse{\Lambda}\cdot [a+b]
  -\frac{1}{2}\,\transpose{[a-b]}\cdot\inverse{\Lambda}\cdot [a-b] \nonumber\\
  &&
  -\frac{1}{2}\,\transpose{[y_1-R\cdot(a+b)-\mu]}\cdot\inverse{C_1}\cdot [y_1-R\cdot(a+b)-\mu] \nonumber\\
  &&
  -\frac{1}{2}\,\transpose{[y_2-R\cdot(a-b)-\mu]}\cdot\inverse{C_2}\cdot [y_2-R\cdot(a-b)-\mu] \nonumber\\
  &&
  + Q \quad ,
\end{eqnarray}
where $Q$ is a constant that doesn't matter to our inference. All we
have to do now is complete the square, do the $a$ integral, and look
at the $b$-dependence of the result.

\end{document}
